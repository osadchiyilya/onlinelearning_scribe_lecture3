\documentclass[11pt]{article}
\usepackage{amsmath,amsfonts,amssymb,amsthm}
\usepackage{times}

\usepackage{geometry}
\geometry{verbose,tmargin=2cm,bmargin=2.5cm,lmargin=2.5cm,rmargin=2.5cm}

\makeatletter

\newtheorem{theorem}{Theorem}[section]
\newtheorem{lemma}[theorem]{Lemma}
\newtheorem{corollary}[theorem]{Corollary}
\newtheorem{claim}[theorem]{Claim}

\theoremstyle{definition}
\newtheorem{definition}[theorem]{Definition}
\theoremstyle{definition}
\newtheorem{example}[theorem]{Example}

\makeatother


%%% fill in details here
\def \lecturenum  {1}
\def \lecturedate {February 4, 2019}
\def \scribe      {John }
%%%

\input{defs}

\begin{document}

%%% make header - do not modify! 
\noindent
\begin{minipage}[t]{1\columnwidth}%
\textsc{Introduction to Online Learning}\hspace*{\fill}48717
\vspace{2mm}

\textsc{\LARGE Lecture \#\lecturenum}\hspace*{\fill}\textsc{\lecturedate}

\noindent \rule[0.5ex]{1\linewidth}{1pt}

\textsc{Lecturer: Kfir Levy\hspace*{\fill}Scribe: \scribe}
\vspace{10mm}
\end{minipage}
%%%





%%%%%%%%%%%%%%%%%%%%%%%%%%%%%%%%%%%%%%%%%%%%%%%%%%%%%%%%%%%%%%%%
%% BODY OF SCRIBE NOTES GOES HERE
%%%%%%%%%%%%%%%%%%%%%%%%%%%%%%%%%%%%%%%%%%%%%%%%%%%%%%%%%%%%%%%%









\section{A Section}

\subsection{A Subsection}

\begin{definition}
A definition. hypothesis class $\H$.
\end{definition}

\begin{lemma}
A lemma.
\end{lemma}

\begin{theorem}
A theorem.
\end{theorem}

\begin{example}
An example.
\end{example}

\section{Notation }

We use the following mathematical notation in this writeup:
\begin{itemize}
\item
$d$-dimensional  Euclidean space is denoted $\reals^d$. 
\item
Vectors are denoted by boldface lower-case letters such as $\x \in \reals^d$.  Coordinates of vectors are denoted by regular brackets $\x(i)$ 
\item
Matrices are denoted by boldface upper-case letters such as $\mathbf{X}  \in \reals^{m \times n}$.  Their coordinates by $\mathbf{X}(i,j)$. 
\item
Functions are denoted by lower case letters $f: \reals^d \mapsto \reals$. 

\item 
The $k$-th differential of function $f$ is denoted by $\nabla^k f \in \reals^{d^k}$.  The gradient is denoted without the superscript, as $\nabla f$. 

\item
We use the mathbb macro for sets, such as $\K \subseteq \reals^d$.  

\end{itemize}



%%%%%%%%%%%%%%%%%%%%%%%%%%%%%%%%%%%%%%%%%%%%%%%%%%%%%%%%%%%%%%%%

\end{document}